\iffalse
% ----------------------------------------------------------------------------------------------------
\subsection*{---------------------}
***general discussion on CBM as usual

***Condition-based maintenance (CBM) has received considerable attention in the literature.
The central idea is to maintain technical systems or components at just the right time,
that is, before they fail,
but not too early, in order to keep reliability high and operating costs low.
%most fully use the component's or system's lifetime and to save on maintenance work

***trade-off between risk of failure during operation
(can lead to costly downtime: idle workforce, missed production, penalties, loss of reputation)
and costs of premature maintenance
(wasting potential component / system lifetime, downtime cost, cost of maintenance work)

***so CBM must be based on some information about the state or health of the component / system.
Two ways: continuous monitoring CBM and inspection-based CBM.

***Continuous monitoring CBM policies are usually derived
using a directly observable continuously measurable condition / degradation signal,
or constructs such a signal or health status using indirect measurements.
Estimated time till failure (RUL) \citep{2014:rul-review, 2011:rul-review-statistical}
via distance of current signal level to a fixed known failure threshold.

***Inspection-based CBM via delay time model, modeling the time between detectable degradation and failure,
again time till failure via distance of current degradation level to fixed known failure threshold.

***In both cases, maintenance decision via control limit / threshold for the signal
(RUL not explicitely calculated)
by minimizing the expected unit cost rate,
which is often approximated using the renewal reward theorem.
%but in practice threshold often not known
\subsection*{---------------------}

Here we propose a different approach to CBM
for the case when no degradation signal for the system is available,
but system components can be identified and their functioning status
(working or not working) can be monitored.
In this situation, one can use the system's layout in reliability terms
(i.e., its reliability block diagram) and information on components' status
to directly calculate the residual life distribution (RLD),
and base the maintenance policy on this distribution.
(In fact, our approach can be seen as CBM based on a multivariate degradation signal,
where each component sends a binary signal, and the reliability block diagram is used
for sensor fusion.)

Our approach can also be seen as a generalization of policies for $k$ out of $N$ systems to arbitrary system layouts.
Furthermore, our approach allows for multiple types of components in the system,
with each component type having its own failure behaviour.

%We use a parametric model not for a degradation signal of the system or for the delay time, but for component lifetimes:
For each component type, we use the well-known Weibull distribution to model component lifetimes.
To keep the model simple and to demonstrate its feasability, we assume the shape parameter for each component type to be known,
focusing on learning of the scale parameter,
and that components fail independently.
These simplifying assumptions will often not hold in practice,
then the model can serve as a first approximation.
Generally, we see our contribution as opening up an entirely new way to derive CBM policies,
and thus consider the model in this paper more as a proof of concept,
where extension to more realistic models must happen in a second step.
As an example, it is possible to extend the model to learn also the shape parameter along the lines of
\cite{1969:soland},
but this is not included here as it would complicate presentation.
%need only to observe when a component fails,
%and expert information about expected component failure times.

The Bayesian approach to the Weibull model \citep[see, e.g.,][]{1996:mazzuchi-soyer} allows to integrate, for each component type,
three sources of information: expert knowledge, data from component tests,
and the status of components in the monitored system. %, i.e., current system condition.
These information sources are integrated into a single component model
using the iterative nature of Bayesian updating,
leading to a so-called posterior predictive distribution for the number of components surviving at any time in the future.

These predictive distributions are then used to calculate the exact distribution of system residual life
using the survival signature \citep{2012:survsign}.
(***we therefore accomplish the same as \cite{2013:si-et-al} for their situation.)

By its Bayesian nature, the system residual life distribution adequately reflects the uncertainties %in RUL estimation
related to modeling and prediction \citep{2015:sankararaman},
readily adapting to any changes in component behaviour.
The use of conjugate priors means that we get explicit formulas for the RLD,
so no numerical integration or simulation techniques are necessary.
This allows for a very frequent, or even real-time, update of the RLD,
taking into account the changed structure when components have failed
and the information gain from updating the component reliability distributions.

Based on the RLD at any current time $\tnow$,
the optimal moment of maintenance given all information available at $\tnow$
is then determined by minimizing the expected one-cycle unit cost rate.
The expected one-cycle unit cost rate, also known as one-cycle criterion
\citep{1984:ansell-bendell-humble,1996:mazzuchi-soyer,2006:coolen-schrijner-coolen},
makes the trade-off between preventive and corrective maintenance according to the current cycle only,
unlike the usually employed renewal based criterion,
which approximates the unit cost rate using a renewal argument \citep[p.~296]{1996:mazzuchi-soyer},
but is not suited for situations where one may whish to change the strategy per cycle \citep{2006:coolen-schrijner-coolen}.

Since the RLD changes with current time $\tnow$,
the corresponding optimal moment of maintenance $\tausnow$ changes as well,
leading to a dynamic adaptive maintenance policy,
similar to a CBM policy based on a continuosly monitored degradation signal.
However, unlike such threshold-based CBM policies,
our approach allows to easily take into account the time needed to set up maintenance work (known as set-up time),
since it gives the optimal moment of maintenance as a time beyond current time $\tnow$,
and maintenance can be initiated as soon as $\tausnow$ equals the set-up time.

In this paper, we assume that at the moment of maintenance,
all components in the system are replaced,
so the cost parameters in the unit cost rate must be determined accordingly.
One could instead consider other replacement schemes (e.g., replacing only the failed components),
or indeed optimize the unit cost rate over all possible replacement schemes,
since our method for RLD calculation can handle differently aged components. 
However, optimizing over replacement schemes opens up a full research programme on its own,
and for situations with high set-up costs,
as is typical for CBM applications,
the replace-all scheme is most likely the cost-optimal strategy. 

***figure of process: component priors $\to$ component posteriors $\to$ system RLD $\to$ $\gnow(\tau)$ $\to$ $\tausnow$

***operation: recalculate $\tausnow$ at fixed dense grid of time points (quasi-continuous),
this makes sense even when no failures happen because the fact that components in the system still function
gives information on component distributions;
alternatively, do the recalculation every time when a component fails (failures contain the most information)

***model allows also to switch to corrective policy when $\gnow(\tau)$ monotonely decreasing (then $\tausnow \to \infty$)

***Expert input in our model is, for each component type,
Weibull shape parameter, expected failure time (translated to scale parameter),
and expert info weight (how sure about mean failure time guess).
Component test data (can include right-censored observations) is optional.
Further input is system layout (reliability block diagram indicating which component belongs to which type)
and the cost parameters $c_p$ and $c_u$.

***here all expressed in time, but could also be in usage (number of cycles, etc).

***compare to $M^*$ out of $N$ policy:
we are dynamic (would mean for a $M^*$ out of $N$ policy that $M^*$ decreases over time due to component aging),
we learn about component lifetimes,
and we allow for set-up time
\fi
% ----------------------------------------------------------------------------------------------------


% ----------------------------------------------------------------------------------------------------
\iffalse
***CBM via RUL/RLD, $M^*$ out of $N$ policies?
***also generally Bayesian methods in maintenance optimization?
***cite CBM review Olde Keizer, Flapper, Teunter, and other refs from it?

Example for Bayesian method in CBM via inspections: \cite{2007:wang-jia}
propose a model that uses Bayesian updating,
but the model aims to identify an optimal inspection interval.
%This is CBM via inspections, not CBM via continuous monitoring as we aim to do.

\begin{scriptsize}
Wang \& Jia (2007) have a model for defects arrival (homogeneous Poisson process with rate $\lambda$, so no aging),
model for delay time (time between defect and failure) is Weibull $h \sim \wei(\alpha,\beta)$,
and each of $\lambda, \alpha, \beta$ has a Gamma prior distribution.
The expected overall cost rate is then used to find a cost-optimal inspection interval.
For this they assume that failures are immediately repaired,
and defects found in an inspection are repaired during inspection.

They use prior predictive distributions for certain statistics (number of defects, number of failures, \ldots between $0$ and $T$)
to obtain prior values for the Gamma hyperparameters,
by solving prior predictive equations -- but give no detail on how they actually do this.

In the end they actually do not advocate to calculate the posterior expected cost rate (15),
but rather plug in posterior estimates for $\lambda, \alpha, \beta$,
so neglecting uncertainty in estimation.
(We do not assume a parametric distribution for delay time,
it rather arises as a consequence of the system layout and component failure distributions.)

\end{scriptsize}

Example for Bayesian method in CBM with continuous monitoring: \cite{2011:elwany-et-al}
(exponential degradation model, whose parameters $\theta'$ and $\beta$ each have a prior
which is updated using the degradation signal history,
and use total expected infinite-horizon discounted cost to determine the maintenance policy,
RUL not explicitely calculated)

Example for Bayesian method in prediction of system remaining useful life: \cite{2012:sun-et-al}
(constructs health index for a system based on sensor measurements,
health status prediction is updated sequentially,
leads to RUL distribution like our method,
but no link from RUL to maintenance decision)

\cite{2013:si-et-al} do CBM with continuous monitoring for a single component,
taking into account the whole degradation path history.
The authors provide exact expressions for the RUL distribution, which is updated in an empirical Bayesian framework using conjugate priors.
The RUL distribution is used to construct a replacement decision model using the unit cost rate via renewal reward
(eq. 35)\\
***this is most similar to what we do: exact RUL distribution $\to$ maintenance decision,
but we have component status instead of degradation signal as basis for RUL,
and we model epistemic uncertainties!\\

\cite{2011:kim-et-al} develop a periodic monitoring CBM policy
where a maintenance decision is triggered when
a Bayesian control chart (a sequentially updated health indicator) %posterior probability that system is in a 'warning' state
exceeds a control limit (threshold) that is determined
by minimizing the expected average cost per time unit.
(We use the same cost criterion, but base the maintenance decision directly on our exact RUL estimation.)

***typical recent example for Bayesian network models in maintenance -- could also cover system reliabilty terrain***
maybe Jones et al (RESS 95:3, 2010) ``The use of Bayesian network modelling for maintenance planning in a manufacturing industry'',
or Bouaziz et al (2013) ``Towards Bayesian network methodology for predicting equipment health factor of complex semiconductor systems''

\begin{scriptsize}
Bayesian networks (BNs) are a very general method to jointly model multiple dependent random variables in a Bayesian way. 
This needs assessments of conditional independece relations, and conditional probability models for each variable,
the latter often in form of conditional probability tables (discrete variables).
A Bayesian network models probabilistic relations between variables,
unlike a reliability block diagram, which gives a deterministic relation
between the status of components and the system state.

One could model a system as a BN (each component is a variable),
but that would add a lot of complications.
Most algorithms for BNs assume discrete distributions,
and do not scale well to large networks (many tasks are NP-hard).

The above papers seem to use a BN to model a degradation signal or (system) health index (have to check again to be sure).
Our approach is different, as we get an exact formulation for the system RUL directly.
\end{scriptsize}
\fi
% ----------------------------------------------------------------------------------------------------

