\documentclass[authoryear]{elsarticle}

% ------------ packages -------------

\usepackage[utf8]{inputenc}
\usepackage[OT1]{fontenc}
\usepackage{graphicx}
\usepackage[english]{babel}

\usepackage{amsmath}
\usepackage{amsfonts}
\usepackage{amssymb}
\usepackage{amsthm}
\usepackage{bm}

\usepackage[usenames,dvipsnames]{xcolor}
\usepackage{booktabs}
\usepackage{tikz}

\usepackage{url}
\usepackage[bookmarks]{hyperref}

%\usetikzlibrary{shapes.misc,fit}
\usetikzlibrary{%
   arrows,%
   calc,%
   fit,%
   patterns,%
   plotmarks,%
   shapes.geometric,%
   shapes.misc,%
   shapes.symbols,%
   shapes.arrows,%
   shapes.callouts,%
   shapes.multipart,%
   shapes.gates.logic.US,%
   shapes.gates.logic.IEC,%
   er,%
   automata,%
   backgrounds,%
   chains,%
   topaths,%
   trees,%
   petri,%
   mindmap,%
   matrix,%
   calendar,%
   folding,%
   fadings,%
   through,%
   patterns,%
   positioning,%
   scopes,%
   decorations.fractals,%
   decorations.shapes,%
   decorations.text,%
   decorations.pathmorphing,%
   decorations.pathreplacing,%
   decorations.footprints,%
   decorations.markings,%
   shadows}

%\usepackage{hyperref}
%\usepackage[bookmarks]{hyperref}
%\usepackage[colorlinks=true,citecolor=red,linkcolor=black]{hyperref}

% ------------ custom defs -------------

\newcommand{\reals}{\mathbb{R}}
\newcommand{\posreals}{\reals_{>0}}
\newcommand{\posrealszero}{\reals_{\ge 0}}
\newcommand{\naturals}{\mathbb{N}}

\newcommand{\dd}{\,\mathrm{d}}

\newcommand{\mbf}[1]{\mathbf{#1}}
\newcommand{\bs}[1]{\boldsymbol{#1}}
\renewcommand{\vec}[1]{{\bm#1}}

\newcommand{\uz}{^{(0)}} % upper zero
\newcommand{\un}{^{(n)}} % upper n
\newcommand{\ui}{^{(i)}} % upper i

\newcommand{\ul}[1]{\underline{#1}}
\newcommand{\ol}[1]{\overline{#1}}

\newcommand{\Tsys}{T_\text{sys}}

\newcommand{\Rsys}{R_\text{sys}}
\newcommand{\lRsys}{\ul{R}_\text{sys}}
\newcommand{\uRsys}{\ol{R}_\text{sys}}

\newcommand{\fsys}{f_\text{sys}}
\newcommand{\Fsys}{F_\text{sys}}
\newcommand{\lFsys}{\ul{F}_\text{sys}}
\newcommand{\uFsys}{\ol{F}_\text{sys}}

\newcommand{\lgt}{\ul{g}}
\newcommand{\ugt}{\ol{g}}

\newcommand{\E}{\operatorname{E}}
\newcommand{\V}{\operatorname{Var}}
\newcommand{\wei}{\operatorname{Wei}} % Weibull Distribution
\newcommand{\ig}{\operatorname{IG}}   % Inverse Gamma Distribution

\newcommand{\El}{\ul{\operatorname{E}}}
\newcommand{\Eu}{\ol{\operatorname{E}}}

\def\yz{y\uz}
\def\yn{y\un}
%\def\yi{y\ui}
\newcommand{\yfun}[1]{y^{({#1})}}
\newcommand{\yfunl}[1]{\ul{y}^{({#1})}}
\newcommand{\yfunu}[1]{\ol{y}^{({#1})}}

\def\ykz{y\uz_k}
\def\ykn{y\un_k}

\def\yzl{\ul{y}\uz}
\def\yzu{\ol{y}\uz}
\def\ynl{\ul{y}\un}
\def\ynu{\ol{y}\un}
\def\yil{\ul{y}\ui}
\def\yiu{\ol{y}\ui}

\def\ykzl{\ul{y}\uz_k}
\def\ykzu{\ol{y}\uz_k}
\def\yknl{\ul{y}\un_k}
\def\yknu{\ol{y}\un_k}

\newcommand{\ykzfun}[1]{y\uz_{#1}}
\newcommand{\ykzlfun}[1]{\ul{y}\uz_{#1}}
\newcommand{\ykzufun}[1]{\ol{y}\uz_{#1}}

\def\nz{n\uz}
\def\nn{n\un}
%\def\ni{n\ui}
\newcommand{\nfun}[1]{n^{({#1})}}
\newcommand{\nfunl}[1]{\ul{n}^{({#1})}}
\newcommand{\nfunu}[1]{\ol{n}^{({#1})}}

\def\nkz{n\uz_k}
\def\nkn{n\un_k}
\newcommand{\nkzfun}[1]{n\uz_{#1}}
\newcommand{\nkzlfun}[1]{\ul{n}\uz_{#1}}
\newcommand{\nkzufun}[1]{\ol{n}\uz_{#1}}

\def\nzl{\ul{n}\uz}
\def\nzu{\ol{n}\uz}
\def\nnl{\ul{n}\un}
\def\nnu{\ol{n}\un}
\def\nil{\ul{n}\ui}
\def\niu{\ol{n}\ui}

\def\nkzl{\ul{n}\uz_k}
\def\nkzu{\ol{n}\uz_k}
\def\nknl{\ul{n}\un_k}
\def\nknu{\ol{n}\un_k}


\def\taut{\tau(\vec{t})}
\def\ttau{\tilde{\tau}}
\def\ttaut{\ttau(\vec{t})}

\def\MZ{\mathcal{M}\uz}
\def\MN{\mathcal{M}\un}

\def\MkZ{\mathcal{M}\uz_k}
\def\MkN{\mathcal{M}\un_k}

\def\PkZ{\Pi\uz_k}
\def\PkN{\Pi\un_k}
\newcommand{\PZi}[1]{\Pi\uz_{#1}}

\def\tnow{t_\text{now}}
\def\tpnow{t^+_\text{now}}


% ------------ options -------------

\allowdisplaybreaks

\journal{***}

\begin{document}

% ------------ frontmatter -------------

\begin{frontmatter}
\title{Condition-Based Maintenance for Systems based on Bayesian P-Box Component Models***}

\author[tue]{Gero Walter}
\ead{g.m.walter@tue.nl}
\author[tue]{Simme Douwe Flapper}
\ead{s.d.p.flapper@tue.nl}

\address[tue]{School of Industrial Engineering, Eindhoven University of Technology, Eindhoven, Netherlands}


\begin{abstract}
CBM when no (continuous) degradation signal for system available, but status of components observable (works yes/no)

use method from asce-asme paper:
gives RUL in form of a p-box (set of distributions), base maintenance decision directly on that
(via upper expected cost rate)

Bayesian approach allows
component models to include expert info, test data (if available), and status of components in the running system

main message: can do CBM for systems without degradation signal,
especially useful for redundant systems,
method reflects aleatory and epistemic uncertainty in all information sources
\end{abstract}

\begin{keyword}
condition-based maintenance \sep system reliability \sep remaining useful life \sep survival signature \sep imprecise probability \sep p-box
\end{keyword}
\end{frontmatter}

% ------------ manuscript -------------

\section{Introduction}
\label{intro}

main message:
can do CBM for systems without degradation signal,
especially useful for redundant systems,
method reflects aleatory and epistemic uncertainty in all information sources

use method from asce-asme paper:
gives RUL in form of a p-box (set of distributions), base maintenance decision directly on that
(via upper expected cost rate?)

Furthermore, our method appropriately reflects epistemic uncertainty,
by using sets of conjugate priors \citep{diss} for the component models.
The resulting sets of component posteriors are equivalent to parametric p-boxes,
an imprecise probability model popular in risk and reliability engineering.

***figure of process: set of component priors $\to$ set of component posteriors $\to$ system RUL $\to$ $g(\tau)$ $\to$ $\tau^*$

***single $\tau^*$ vs set of $\tau^*$: decision support

***discuss \cite{2015:sankararaman}, epistemic vs. aleatoric uncertainty: big topic in reliability and risk

***refer to Terje Aven, NASA Langley UQ challenge, illustration that uniform distribution does not express ignorance (Edoardo Patelli)***

***we go beyond what \cite{2015:sankararaman} recommends

The proposed system RUL model adequately reflects uncertainty in the RUL distribution
according to the current system status,
in particular making more cautious predictions when failure behaviour in the monitored system
seems to deviate from what is expected from expert knowledge and previous (test) data.

Expert input in our model is, for each component type,
Weibull shape parameter, range of expected failure time (translated to scale parameter),
and range of expert info weight (how sure about mean failure time guess).
Component test data (can include right-censored observations) is optional.
By using ranges we do a systematic sensitivity analysis.

Our model gives lower and upper bounds for the distribution of time till system failure,
taking current (at time $\tnow$) system condition into account.

An advantage of our approach is that
it indicates, for any current time $\tnow$, at which point in the future it is economical to repair the system,
thus allowing for set-up time for maintenance work.

***all expressed in time, but could also be in usage (number of cycles, etc).

\section{***summary of ASCE-ASME paper method}

sets of priors on component survival,
survival signature,
test data inclusion
(write down explicit formula for test data including censored observations),
noninformatiove censoring,
set of system reliability curves as output,
wider set in case of prior-data conflict

\section{***extensions to ASCE-ASME paper method}
\label{sec:extensions}


output is a set of current system reliability curves
taking into account the current system state,
including the current ages of system components, 
and the lifetime histories of all component types,
including test data and expert assessments.
The set of current, updated system reliability curves gives a RUL estimate
in the form of a p-box, thus reflecting both aleatoric and epistemic uncertainty in the estimate.

\section{***from (set of) reliability curves to maintenance decision}

To find the optimal moment for maintenance based on the current set of system reliability curves,
we minimize the expected average cost per time unit,
or unit cost rate, $g(\tau)$ %, where $\tau$ is the moment when preventive maintenance is exectued.
(could be discounted as well, out of scope here.)

In the literature, commonly two ways to calculate the unit cost rate are considered,
a renewal reward theory based long-term unit cost rate $g_r(\tau)$,
and the one-cycle cost rate $g_1(\tau)$ \citep{1996:mazzuchi-soyer}.
We will discuss both approaches here with their advantages and disadvanatges.

Let $c_p$ be the cost of planned / preventive maintenance, and $c_u$ the cost of unplanned / breakdown maintenance, where $c_p < c_u$.

\subsection{***unit cost rate via renewal reward}

The renewal reward theory based approach considers the unit cost rate as a long-term average,
where the proposed policy is used over a large number of replacement cycles,
and cycles can be seen as stochastic copies of each other.
Under this assumption, the unit cost rate is, according to renewal theory,
given as the expected cycle cost $ECC(\tau)$ divided by the expected cycle length $ECL(\tau)$:
\begin{align}
g_r(\tau) &= \frac{ECC(\tau)}{ECL(\tau)} = \frac{c_p \Rsys(\tau) + c_u \big(1-\Rsys(\tau)\big)}{\tau \Rsys(\tau) + \int_0^\tau \Rsys(t) \dd t}\,,
\end{align}
and $\tau_r^* = \arg\min g_r(\tau) - \tnow$ is the cost-optimal repair time from now
(since $\Rsys(\cdot)$ is always in terms of time since system startup)

***most often used in literature \citep{2013:si-et-al}, \citep{2011:kim-et-al}

***stochastic copies assumption implies that the system is in the same state at the start of each cycle,
so $c_p$ needs to include the costs of replacing all components
to bring the system into the same state as at startup time.

\subsection{***one-cycle unit cost rate}

The one-cycle unit cost rate approach considers only the costs with respect to the current cycle.
Conditional on the (random) system failure time $\Tsys$, the unit cost rate is 
\begin{align*}
g_1(\tau \mid \Tsys) &=
\begin{cases}
c_p / \tau  & \text{if } \Tsys \ge \tau \\
c_u / \Tsys & \text{if } \Tsys < \tau
\end{cases}
\end{align*}

Taking the expectation over $\Tsys$ leads to the unit cost rate being
\begin{align*}
g_1(\tau) &= \E[g_1(\tau \mid \Tsys)] = \frac{c_p}{\tau} \Rsys(\tau) + c_u \int_0^\tau \frac{1}{t} \fsys(t) \dd t
\end{align*}
***here, $\Rsys(\cdot)$ shifted such that $\tnow = 0$.

Due to our extensions from Section~\ref{sec:extensions},
maintenance work can consist of replacing (or returning to an as-good-as-new state) some or all failed or non-failed components,
as arbitrary component ages can be taken into account.
(However, computation times decrease when all components of the same type are replaced at the same time,
as then the duplication of component types for the survival signature decomposition is not necessary.)


\subsection{***bounds for unit cost rate}

Bounds for $\Rsys(t)$ lead to bounds for $g(\tau)$.
%For each $\tau$, fix $\Rsys(\tau) \in [\lRsys(\tau), \uRsys(\tau)]$ and optimize $I(\tau) = \int_0^\tau \Rsys(t) \dd t$
%nonparametrically: $\ol{I}(\tau)$ by $\uRsys(t)$, $\ul{I}(\tau)$ by $\lRsys(t)$ until level of $\Rsys(\tau)$.
Can get bounds $\lgt(\tau)$ and $\ugt(\tau)$ in the same way as bounds $\lRsys(t)$ and $\uRsys(t)$ in ASCE-ASME paper:
need to optimize only over
$\nkzfun{1} \in \left[\nkzlfun{1}, \nkzufun{1}\right], \ldots, \nkzfun{K} \in \left[\nkzlfun{K}, \nkzufun{K}\right]$.

This works because, like $\Rsys(t)$, also $g(\tau)$ is monotone in $\ykzfun{1}, \ldots \ykzfun{K}$:
$\Rsys(\tau)$ is monotone in $\ykzfun{1}, \ldots \ykzfun{K}$ (ASCE-ASME paper),
and $g(\tau)$ is monotonely decreasing in $\Rsys(\tau)$:

First look at $I(\tau) := \int_0^\tau \Rsys(t) \dd t$.
A lower bound for $I(\tau)$ is $\tau \Rsys(\tau)$ (obtained when $\Rsys(t)$ drops to $\Rsys(\tau)$ immediately after $t=0$),
and an upper bound for $I(\tau)$ is $\tau$ (obtained when $\Rsys(t)$ remains at $1$ until immediately prior to $t = \tau$).
Therefore, for each $\tau$ and $\Rsys(\tau)$ there exists a constant $R^* \in [\Rsys(\tau), 1]$
such that $I(\tau) = \tau R^*$.
Note that $R^*$ increases when $\Rsys(\tau)$ increases.

Setting $\Rsys(\tau) =: R$ and $g(\tau) =: g$ to simplify notation, we get
\begin{align*}
g &= \frac{c_p R + c_u (1-R)}{\tau R + \tau R^*}
   = \frac{(c_p - c_u) R + c_u}{\tau R + \tau R^*}\,.
\end{align*}
As $(c_p -c_u) < 0$, the enumerator of $g$ is monotonely decreasing in $R$.
The denominator of $g$ is monotonely increasing in $R$, since $R^*$ must increase when $R$ increases.
In total, $g$ is thus monotonely decreasing in $R$.

***formulate above as theorem?


\subsection{***from bounds for $g(\tau)$ to (set of) $\tau^*$:}

%worst case, best case, interval (how?)
\begin{itemize}
\item worst case (minimax strategy): minimizing $\ugt(\tau)$ leads to $\tau^*_u$.
\item interval dominance: for each $\tau$, we get a $g(\tau)$ interval $[\lgt(\tau), \ugt(\tau)]$.
Throw out all $\tau$'s for which $\lgt(\tau) \ge \ugt(\tau^*_u)$.
This will leave us with the set of $\tau$'s that are not dominated by $\tau^*_u$.
(The set will be quite wide probably.)
\item parametric / maximality (?): collect all $\tau$'s for which there is a prior parameter combination such that $\tau$ minimizes $g(\tau)$.
(Need to do grid search over $\nkz$'s, or keep optimal pars when calculating $\lgt(\tau)$.)
\end{itemize}

\subsection{***simple 95\% policy}

Or, alternatively to all the above, simple 95\% policy?
Repair time from now such that lower system reliability is at least 95\%, i.e., upper system failure probability is at most 5\%.

\section*{Acknowledgements}

CAMPI

Frank Coolen for inspiring discussions

\section*{Bibliography}

\bibliographystyle{elsarticle-harv}

\bibliography{refs}

\end{document}
